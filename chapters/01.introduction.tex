This chapter serves as a backgrounder for readers to have an overview of the study even without prior reference to other publications on the topic. The introduction is the first chapter of the thesis and must include the objective/s and justification of the study as well as the limitations set by the proponent. The introduction is the proper place to define any specialized terms and concepts used in the thesis.
		
\section{Project Context}
This area refer to the discussion of the project itself with the inclusion on general situation in terms of its exigency. 

\section{Purpose and Description}
The statement of the problem is the backbone of the proposal/paper. This is the main idea of the entire research project.  This is a statement that you can prove with evidence/s. Well-constructed problem statements will convince your audience that the problem is real and worth having you investigates. Well-constructed problem statement defines the problem and helps identify the variables that will be investigated in the study.

\section{Objective of the Study} 
This section summarizes what is to be achieved by the study. This usually contains general and specific objectives.  Research objectives are closely related to research problem.

\section{Significance of the Study} 
This section describes or explains the potential value of the study and findings. It should be clear in here, the target audience for the study and how the results will be beneficial for them. It answer the questions – Why is it important? To whom it will be beneficial?

\section{Scope and Limitation} 
This section sets parameters of the study. Limitations are the inherent problems encountered by the researcher, thus, stating the limitations of the study can be very useful for readers in interpreting the results of the study.
